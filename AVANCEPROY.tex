\documentclass[conference]{IEEEtran}
\usepackage{url}
\usepackage{graphicx}

\begin{document}

\title{Tecnologías de la información en el Desarrollo de Software Multiplataforma}

\author{
\IEEEauthorblockN{
Armenta Telles Jesús Manuel 0321101244,
Contreras Rangel Martin 0322103695, \\
Diaz Escalante José Ángel 0322103701,
Higuera Sánchez Dulce Mariela 0322103734, \\
Reyes Contreras Ramsés 0322103800,
Rodríguez Cacho Ximena Charleene 0322103808
}
\IEEEauthorblockA{Grupo: 4-A, Turno: Matutino \\
Materia: Estándares y métricas para el desarrollo de software \\
Docente: Lara Lagunes Marylu \\
Tijuana, B.C a 02 de febrero de 2024}
}

\maketitle

\section{ÍNDICE}

\begin{enumerate}
  \item Introducción
  \item Planteamiento del problema
  \item Descripción por materia
  \begin{itemize}
    \item Estándares y métricas para el desarrollo de software
    \item Principios de IOT
    \item Diseño de apps
    \item Aplicaciones web orientada a servicios
    \item Evaluación y mejora para desarrollo de software
  \end{itemize}
  \item Cronograma de actividades
\end{enumerate}

\section{INTRODUCCIÓN}

El documento aborda puntos fundamentales para proponer el problema de la industria, cuyo enfoque es solucionar la administración de la producción de vestidos de gala, al ser una compañía que los elabora y vende, se busca implementar una alternativa de solución para controlar los insumos, y una interacción del usuario intuitiva y óptima para que el cliente solicite el producto. Se plantea un análisis que nos permite partir desde la identificación del problema, definiendo la raíz del problema de la empresa con detalle, seguido de un diagnóstico que, pone en orden los diversos inconvenientes de la empresa, para a partir de observaciones y datos concretos, brindar una valoración precisa del problema. Cuenta con un apartado donde se describe el impacto de las materias cursadas este cuatrimestre y como se verán reflejadas en el proyecto, justificando cada una y resaltando su relevancia, así como su impacto funcional. Abordando como materias del proyecto integrador: Estándares y métricas para el desarrollo de software, junto con su evaluación y mejora, se documenta cómo abordaron los problemas en su etapa base, y que mejoras se pretenden implementar para hacer el producto más escalable, poniendo en práctica el conocimiento de las tecnologías aprendidas hasta el momento, siguiendo los estándares de calidad para el desarrollo del proyecto. Se tendrán dos productos: una página web y una aplicación en Android, donde se implementarán nuevas funciones de servicios que permitan facilitar la experiencia de usuario, viéndose involucradas las asignaturas de aplicaciones web orientada a servicios y diseño de aplicaciones. Por último, se contará con un inicio de sesión para el administrador de la página, que será mediante reconocimiento biométrico, mediante lector de huella, haciéndolo un proyecto integrador innovador que brinde una solución tecnológica.

\section{PLANTEAMIENTO DEL PROBLEMA}

Dentro de la empresa, se tienen problemáticas con respecto a la capacidad para operar de manera eficiente en el sector de producción y distribución de vestidos de gala. En el área de producción, se observa un uso ineficiente de los recursos debido a la falta de herramientas tecnológicas adecuadas. No se cuenta con un inventario digital optimizado de materia prima, así como la distribución para las tres sucursales que se tienen, por lo que esto genera pérdidas y retrasos en la producción, tanto monetarias como en insumos. En cuanto a la logística, la empresa experimenta dificultades en la distribución de los vestidos a los proveedores. La falta de un sistema de gestión de entregas provoca inconsistencias en la asignación de productos y retrasos en las entregas, lo que afecta la relación con los proveedores y genera pérdidas financieras. Además, la ausencia de registros digitales dificulta el seguimiento de las entregas y la identificación de posibles problemas en el proceso. Por último, la base de datos de la empresa carece de medidas adecuadas de seguridad, que bien es funcional, le hace falta medidas para proteger la información en un nivel de encapsulamiento mayor, por lo que se expone a posibles violaciones de datos y pérdidas de información confidencial. La falta de restricciones de acceso y la vulnerabilidad ante el acceso no autorizado plantean riesgos significativos y, por ende, la página web se encuentra vulnerable.

\section{DESCRIPCIÓN POR MATERIA}

A continuación, se describe la justificación del desarrollo del proyecto de acuerdo a cada materia que forma parte del proyecto integrador:

\subsection{Estándares y métricas para el desarrollo de software}

La materia de Estándares y métricas para el desarrollo de software proporciona los fundamentos necesarios para garantizar la calidad y eficiencia en la producción de software. En el contexto de este proyecto, se utilizarán estos estándares y métricas como guía para desarrollar tanto la página web como la aplicación móvil. Esto implica seguir las mejores prácticas establecidas en la industria para la gestión de procesos, la documentación y la calidad del código, asegurando así un producto final robusto y de alto rendimiento.

\subsection{Principios de IOT}

Dado que el proyecto involucra la implementación de una tecnología, decidimos optar por la seguridad biométrica para el inicio de sesión del administrador mediante reconocimiento biométrico a través de un lector de huella, los principios de IoT serán fundamentales. Esto incluirá aspectos como la conectividad, la seguridad y la compatibilidad entre dispositivos, asegurando una integración efectiva y segura de la tecnología biométrica en el sistema.

\subsection{Diseño de apps}

En esta etapa, se abordarán los siguientes aspectos:
\begin{itemize}
  \item SRS del cliente: Se elaborará el documento de Requerimientos de Software (SRS, por sus siglas en inglés) que detallará las necesidades y expectativas del cliente en cuanto a la funcionalidad y características del sistema.
  \item Modelado del cliente: Se realizará un modelado detallado que represente de manera visual la interacción del usuario con la página web y la aplicación móvil, asegurando una experiencia de usuario intuitiva y eficiente.
  \item La maquetación del cliente: Se desarrollarán los diseños de la interfaz de usuario tanto para la página web como para la aplicación móvil, teniendo en cuenta los principios de diseño centrado en el usuario y la usabilidad.
  \item Creación de la aplicación móvil: Se llevará a cabo el diseño y desarrollo de la aplicación móvil, asegurando una integración coherente con la página web y una experiencia de usuario consistente en ambas plataformas.
\end{itemize}

\subsection{Aplicaciones web orientada a servicios}

Esta materia se enfocará en la implementación de APIs (Interfaces de Programación de Aplicaciones) que permitan la comunicación entre diferentes componentes del sistema, así como la integración con servicios externos. Esto facilitará la creación de funcionalidades adicionales en el sistema, como la gestión de pedidos y la interacción con bases de datos.

\subsection{Evaluación y mejora para desarrollo de software}

Se llevará a cabo una evaluación continua del proceso de desarrollo de software, identificando áreas de mejora y aplicando ajustes conforme avance el proyecto. Esto garantizará que el producto final cumpla con los estándares de calidad establecidos, y se optimice tanto en términos de funcionalidad como de eficiencia en el desarrollo.

\section{CRONOGRAMA}

El cronograma estará sujeto a cambios hasta su aprobación.

\begin{figure}[htbp]
  \centering
  \includegraphics[width=1\linewidth]{cronograma.jpeg}
  \caption{Cronograma de actividades}
  \label{fig:cronograma}
\end{figure}

\section{FUENTES DE CONSULTA}

\begin{thebibliography}{99}

\bibitem{li_ecommerce_security}
Li, X. (2014). E-commerce security issues and solutions. International Journal of Computer Applications, 101(15).

\bibitem{welling_php_mysql}
Welling, L., \& Thomson, L. (2016). PHP and MySQL Web Development. Addison-Wesley Professional.

\bibitem{chaffey_digital_marketing}
Chaffey, D., \& Ellis-Chadwick, F. (2019). Digital Marketing. Pearson UK.

\bibitem{dubois_mysql}
Dubois, P. (2008). MySQL. Pearson Education.

\bibitem{kothari_ecommerce_online_shopping}
Kothari, R., \& Kothari, M. (2013). E-Commerce and Online Shopping. International Journal of Computer Applications, 73(15).

\end{thebibliography}

\end{document}
